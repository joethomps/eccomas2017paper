\documentclass{beamer}
 
\mode<presentation>
{
  \usetheme{Boadilla}
  \setbeamercovered{transparent}
  % or whatever (possibly just delete it)
}

\usepackage[english]{babel}
\usepackage[utf8]{inputenc}
\usepackage{graphicx}
\usepackage{times}
\usepackage[T1]{fontenc}
\usepackage{amsmath,amsfonts,amssymb,amsthm,amsbsy,amsmath}
\usepackage{pgfplots}
\pgfplotsset{compat=1.11} 
\usepackage{subfigure}
\usetikzlibrary{shapes,arrows,scopes,patterns,decorations.pathreplacing}
\usepackage[backend=biber, style=authoryear]{biblatex}
\bibliography{eccomas_2017}

\title[Equivalent Mass-Spring Models]{Equivalent Mass-Spring Models of Multibody Spacecraft for the Application of Wave-based Control}

\author{J.W.~Thompson \and W.J.~O'Connor}

\institute[UCD]{University College Dublin}

\date[MBD 2017] % (optional, should be abbreviation of conference name)
{8th Eccomas Thematic Conference on Multibody Dynamics}

\subject{Multibody Dynamics}

\pgfdeclareimage[height=1cm]{university-logo}{ucd_brandmark_colour}
\logo{\pgfuseimage{university-logo}}

\begin{document}

\begin{frame}
  \titlepage
\end{frame}
\begin{frame}{Outline}
  \tableofcontents
\end{frame}

%%%%%%%%%%%%%%%%%%%%%%%%%%%%%%%%%%%%%%%%%%%%%%%%%%%%%%%%%%%%%%%%%%%%%%%%%%%%
%%%%%%%%%%%%%%%%%%%%%%%%%-------MOTIVATION--------%%%%%%%%%%%%%%%%%%%%%%%%%%
%%%%%%%%%%%%%%%%%%%%%%%%%%%%%%%%%%%%%%%%%%%%%%%%%%%%%%%%%%%%%%%%%%%%%%%%%%%%
\section{Motivation}

\subsection{Wave-based Modelling and Control of Lumped Flexible Systems}

\begin{frame}{Wave-based Modelling of Lumped Systems}
  \begin{itemize}
  \item
   	Classic underactuated control problem
  \item
    Exhibits wave like behaviour
  \item
  	Novel approach to control
  \end{itemize}
\end{frame}

\begin{frame}{Wave-based Control System}
  \begin{itemize}
  \item
    Launch and return wave
  \item
    Measure at actuator
    This so-called “wave-based control” (WBC) has been applied successfully to uniform and non-uniform in-line mass-spring systems \footcite{OConnor1998,OConnor2011}.
The aim of this paper is to extend the WBC theory to a wider class of systems, and in particular the multi-body spacecraft described above.
  \end{itemize}
      Problem: there is a specific form needed for WBC
    string of masses and springs
    SISO system: Input is force on first mass, Output is position of first mass
\end{frame}

\subsection{Attitude Control of Multibody Spacecraft}

\begin{frame}{Inline Mass Spring Structure}
\begin{figure}
\includegraphics[width=0.4\textwidth]{images/satellite.jpg}
\caption{Satellite with flexible appendages}
\end{figure}
\end{frame}

\begin{frame}{Inline Mass Spring Structure}
Some example of multibody spacecraft
\begin{columns}
\column{0.5\textwidth}
\begin{figure}
\includegraphics[width=\textwidth]{images/satellite.jpg}
\caption{Satellite with flexible appendages}
\end{figure}
\column{0.5\textwidth}
Some info
\end{columns}
\end{frame}

%%%%%%%%%%%%%%%%%%%%%%%%%%%%%%%%%%%%%%%%%%%%%%%%%%%%%%%%%%%%%%%%%%%%%%%%%%%%
%%%%%%%%%%%%%%%%%%%%%%%%%-------PROBLEM STATEMENT--------%%%%%%%%%%%%%%%%%%%
%%%%%%%%%%%%%%%%%%%%%%%%%%%%%%%%%%%%%%%%%%%%%%%%%%%%%%%%%%%%%%%%%%%%%%%%%%%%
\section{Problem Statement}



%%%%%%%%%%%%%%%%%%%%%%%%%%%%%%%%%%%%%%%%%%%%%%%%%%%%%%%%%%%%%%%%%%%%%%%%%
%%%%%%%%%%%%%%%%%%%%%%%%%-------METHODS--------%%%%%%%%%%%%%%%%%%%%%%%%%%
%%%%%%%%%%%%%%%%%%%%%%%%%%%%%%%%%%%%%%%%%%%%%%%%%%%%%%%%%%%%%%%%%%%%%%%%%
\section{Methods}
\subsection{Inverse Eigenvalue Problem for Jacobi Matrices}
\subsection{Mass Spring Reconstruction}
\subsection{Necessary and Sufficient Conditions for Transformation}


%%%%%%%%%%%%%%%%%%%%%%%%%%%%%%%%%%%%%%%%%%%%%%%%%%%%%%%%%%%%%%%%%%%%%%%%%
%%%%%%%%%%%%%%%%%%%%%%%%%-------RESULTS--------%%%%%%%%%%%%%%%%%%%%%%%%%%
%%%%%%%%%%%%%%%%%%%%%%%%%%%%%%%%%%%%%%%%%%%%%%%%%%%%%%%%%%%%%%%%%%%%%%%%%
\section{Results}

\subsection{Segmented Rocket Model}

\subsection{Test Cases for different sensor-actuator combinations}

\begin{frame}{Test Case 1}
\begin{center}
    \input{graphics/testcase3.tex}
\end{center}
\end{frame}

\begin{frame}{Test Case 2}
\begin{center}
    \input{graphics/testcase1.tex}
\end{center}
\end{frame}

\begin{frame}{Test Case 3}
\begin{center}
    \input{graphics/testcase2.tex}
\end{center}
\end{frame}

\subsection{Basic Ideas for Proofs/Implementation}


%%%%%%%%%%%%%%%%%%%%%%%%%%%%%%%%%%%%%%%%%%%%%%%%%%%%%%%%%%%%%%%%%%%%%%%%%%%%
%%%%%%%%%%%%%%%%%%%%%%%%%-------SUMMARY--------%%%%%%%%%%%%%%%%%%%%%%%%%%
%%%%%%%%%%%%%%%%%%%%%%%%%%%%%%%%%%%%%%%%%%%%%%%%%%%%%%%%%%%%%%%%%%%%%%%%%%%%
\section{Summary}

\begin{frame}{Summary}

mass-spring representations of a wide class of SISO systems.
force on the first mass and the output is the position of the first mass.
These are the type of systems for which WBC has been shown to work well.
The transformation is possible for a system described by Eqs. \ref{eq:modal1}-\ref{eq:modal3}, with non-negative and distinct eigenvalues and corresponding entries in the $\mathbf{b}$ and $\mathbf{c}$ vectors either all having the same sign or all having opposite signs.

It is clear that the non-uniformity of the resulting mass-spring system and the position of this non-uniformity are key in assessing the difficulty of the control problem.
From a wave based perspective large masses in the centre of the cascade lead to the trapping of waves in the outer part of the system increasing settling times.

  % Keep the summary *very short*.
  \begin{itemize}
  \item
    The \alert{first main message} of your talk in one or two lines.
  \item
    The \alert{second main message} of your talk in one or two lines.
  \item
    Perhaps a \alert{third message}, but not more than that.
  \item
    Future Work
    \begin{itemize}
    \item A possible area for future research is to investigate equivalent systems where the input and output may be on different masses internal to the system.
    \item There is also scope for using similar techniques to optimise actuator and sensor locations within a system.
    \end{itemize}
    
  \end{itemize}
\end{frame}



% All of the following is optional and typically not needed. 
\appendix
\section<presentation>*{\appendixname}
\subsection<presentation>*{For Further Reading}

\begin{frame}[allowframebreaks]
  \frametitle<presentation>{For Further Reading}
\printbibliography
\end{frame}

\end{document}
