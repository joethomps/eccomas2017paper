\documentclass{beamer}
 
\mode<presentation>
{
  \usetheme{Boadilla}
  %\usetheme{default}
  %\usetheme{Darmstadt}
  \setbeamercovered{transparent}
  % or whatever (possibly just delete it)
}

\usepackage[english]{babel}
\usepackage[utf8]{inputenc}
\usepackage{graphicx}
\usepackage{times}
\usepackage[T1]{fontenc}
\usepackage{amsmath,amsfonts,amssymb,amsthm,amsbsy,amsmath}
\usepackage{pgfplots}
\pgfplotsset{compat=1.11} 
\usepackage{subfigure}
\usetikzlibrary{shapes,arrows,scopes,patterns,decorations.pathreplacing}
\usepackage[backend=biber, style=authoryear, url=false, doi=false, isbn=false]{biblatex}
\bibliography{eccomas_2017}
\usefonttheme{professionalfonts}

\title[Equivalent Mass-Spring Models]{Equivalent Mass-Spring Models of Multibody Spacecraft for the Application of Wave-based Control}

\author{Joseph Thompson} %\and W. J.~O'Connor}

\institute[UCD, Ireland]{University College Dublin, Ireland}

\date[ECCOMAS MBD 2017] % (optional, should be abbreviation of conference name)
{8th Eccomas Thematic Conference on Multibody Dynamics \\ \vspace{8pt} \small{Czech Technical University, Prague, 19-22 June, 2017}}

\subject{Multibody Dynamics}

\pgfdeclareimage[height=1cm]{university-logo}{ucd_brandmark_colour}
\logo{\pgfuseimage{university-logo}}

\begin{document}

\begin{frame}
  \titlepage
\end{frame}
\begin{frame}{Outline}
  \tableofcontents
\end{frame}

%%%%%%%%%%%%%%%%%%%%%%%%%%%%%%%%%%%%%%%%%%%%%%%%%%%%%%%%%%%%%%%%%%%%%%%%%%%%
%%%%%%%%%%%%%%%%%%%%%%%%%-------MOTIVATION--------%%%%%%%%%%%%%%%%%%%%%%%%%%
%%%%%%%%%%%%%%%%%%%%%%%%%%%%%%%%%%%%%%%%%%%%%%%%%%%%%%%%%%%%%%%%%%%%%%%%%%%%
\section{Motivation}

\subsection{Wave-based Modelling and Control of Lumped Flexible Systems}
\begin{frame}{Underactuated control problem}
\begin{center}
\includegraphics[]{images/lumped_system.pdf}
\end{center}
\begin{itemize}
\item Classic underactuated control problem
\item Rest-to-rest motion
\item Combine gross motion control and active vibration damping
\item With a single actuator
\end{itemize}
\begin{center}
\includegraphics[width=0.85\textwidth]{images/under_actuated_systems.png}
\end{center}
\end{frame}

\begin{frame}{Wave-based Modelling of Lumped Systems}
\begin{itemize}
\item Novel approach - wave based model \footcite{OConnor2011}
\item Decompose the system into two components travelling into and out of the system
\item Wave-transfer function:
$ \quad G(s) =1+ \frac{m s^2}{2k} \pm \sqrt{\frac{m s^2}{k}+\frac{m^2 s^4}{4 k^2}}$
\item Dealing with non-uniformity
\end{itemize}
\includegraphics[width=\textwidth]{images/series_model_tf_unif.pdf}
\end{frame}

\begin{frame}{Wave-based Control System}
%Styles
\tikzstyle{block} = [draw, thick, rectangle, minimum height=3em, minimum width=3em]
\tikzstyle{math} = [draw, thick, circle, minimum size=0.6cm,node distance=2cm, path picture={ 
      \draw[line] (path picture bounding box.south west) -- (path picture bounding box.north east);
      \draw[line] (path picture bounding box.north west) -- (path picture bounding box.south east);}]
\tikzstyle{gain} = [draw, thick, isosceles triangle, minimum size=0.6cm,node distance=1.5cm]
\tikzstyle{pt} = [coordinate]
\tikzstyle{line} = [->,thick]
\tikzstyle{ground}=[color=gray,postaction={draw,decorate,decoration={border,angle=-45,amplitude=0.2cm,segment length=2mm}}]
\tikzstyle{actuator} = [draw=blue!50!black ,fill=blue!50!black!50!white, thick, rectangle, inner sep=0,minimum height=0.6cm, minimum width=0.2cm, node distance=1.8cm]
\tikzstyle{spring} = [thick,black,decorate,decoration={snake,amplitude=3,segment length=10}]
\tikzstyle{wheel} = [thick,orange,decorate,decoration={coil,aspect=0.7,amplitude=5}]
\tikzstyle{cart} = [rectangle, inner sep=0,minimum height=0.8cm, minimum width=1cm, node distance=1.8cm, path picture={ 
      \shadedraw[left color=white,right color=gray!40!white, thick] ([yshift=0.12cm, xshift=0.3pt] path picture bounding box.south west) rectangle ([xshift=-0.3pt,yshift=-0.3pt] path picture bounding box.north east); 
      \draw[very thick, fill=white] ([yshift=0.12cm, xshift=-0.3cm] path picture bounding box.south) circle (0.1cm);
      \draw[very thick,fill=white] ([yshift=0.12cm, xshift=0.3cm] path picture bounding box.south) circle (0.1cm);}]
\tikzstyle{pt} = [coordinate]
\tikzstyle{force}=[->,thick,>=latex,draw=blue,fill=blue]

\begin{tikzpicture}[auto, scale=0.75]

%Nodes
	\node[pt]	at (0,0)	(ref)													{};
	\node[block] 		(con) 		[right of=ref, node distance=1.5cm, align=center] 		{\small Wave-based \\ \small Controller};
	\node[block] 		(act) 		[right of=con, node distance=2.6cm, align=center]  		{\small Actuator \\ \small Dynamics};
	\node[pt]			(for)		[right of=act, node distance=1.4cm]							{};

	
	\node[cart]	(m1)		[right of=for, node distance=1cm]			{$m_1$};
	\node[cart] 	(m2) 		[right of=m1] 	{$m_2$};
	\node[pt] 	(pt1) 		[right of=m2, node distance=1cm] 	{};
	\node[pt] 	(pt2) 		[right of=pt1, node distance=0.5cm] 	{};
	\node[cart] 	(m3) 		[right of=pt2, node distance=1cm] 	{$m_n$};
 	
 	\node[pt]			(int)		[below of=m1, node distance=2cm]							{};
 	\node[pt]			(int2)		[below of=for, node distance=1.5cm]							{};
 	
%Connections	
      \draw[force] (act) -- node {$f_0$} (m1);
 	\draw[line] (ref) -- node {$r$}  (con);
 	\draw[line] (con) -- node {$f_r$} (act);
	\draw[line] (m1) -- node[near start] {$x_1$} (int) -| (con.240);
	\draw[line] (for) -- node[near start] {$f_0$} (int2) -| (con.300);
% 	\draw[line] (minus_x0) -- (k0);
% 	\draw[line] (k0) -- node[] (N2) {$M_{ref}$} (m1);
	
	\draw[spring] (m1) -- node[above] {$k_1$} (m2);
	\draw[spring] (m2) -- node[above] {$k_2$} (pt1);
	\draw[thick, dashed] (pt1) -- (pt2);
	\draw[spring] (pt2) -- node[above] {$k_3$} (m3);
	%\draw[ground] (m1.south west) -- (m3.south east);

\end{tikzpicture}
\begin{itemize}
\item Strategy:
	\begin{enumerate} %%%% copy from eccomas pres
	\item Actuator launches a wave into the system which travels to the right 
	\item Wave reaches the system boundary and is reflected back leftwards
	\item Returning wave is measured and absorbed at the actuator
	\end{enumerate}
\item Intuitive way to do control, respects the delay inherent in the system - with many desirable properties
\item Successfully applied to rectilinear mass-spring systems \footcite{OConnor1998}
\item Objective: extend to a wider class of systems (multi-body spacecraft)
\end{itemize}
\end{frame}

\subsection{Attitude Control of Multibody Spacecraft}
\begin{frame}{Spacecraft Modelled as Multibody Systems}
%%%% images of multi-body spacecraft
\begin{columns}
\column{0.3\textwidth}
\includegraphics[width=\textwidth]{images/satellite.jpg}
\column{0.1\textwidth}
\includegraphics[width=\textwidth]{images/esca.png}
\column{0.6\textwidth}
\includegraphics[width=\textwidth]{images/rocket3.pdf}
\end{columns}
\begin{itemize}
\item How similar are these to the mass spring systems shown above
\item Choice of actuators: TVC, thrusters, momentum wheels
\item Choice of sensor position: IMU, gyroscopes, accelerometers
\end{itemize}
\end{frame}

%%%%%%%%%%%%%%%%%%%%%%%%%%%%%%%%%%%%%%%%%%%%%%%%%%%%%%%%%%%%%%%%%%%%%%%%%%%%
%%%%%%%%%%%%%%%%%%%%%%%%%-------PROBLEM STATEMENT--------%%%%%%%%%%%%%%%%%%%
%%%%%%%%%%%%%%%%%%%%%%%%%%%%%%%%%%%%%%%%%%%%%%%%%%%%%%%%%%%%%%%%%%%%%%%%%%%%
\section{Problem Statement}
\begin{frame}{Problem Statement}
Given a SISO (single-input single-output) undamped system described by:
\begin{equation}
\ddot{\mathbf{q}}(t) + \Lambda\mathbf{q}(t) = \mathbf{b}u(t)
\label{eq:modal1}
\end{equation}
\begin{equation}
y(t) = \mathbf{c}^T \mathbf{q}(t)
\label{eq:modal2}
\end{equation}
$$
\Lambda = \begin{bmatrix}
\lambda_1  &  0 & \cdots & 0 \\
0 & \lambda_2  & \ddots & 0 \\
\vdots & \ddots & \ddots & \vdots \\
0 & 0 & \cdots & \lambda_n \end{bmatrix}
,\quad \mathbf{b} = \begin{bmatrix} b_1 \\ b_2 \\ \vdots \\ b_n \end{bmatrix}
,\quad \mathbf{c} = \begin{bmatrix} c_1 \\ c_2 \\ \vdots \\ c_n \end{bmatrix}
\label{eq:modal3}
$$
where $u(t)$ is the input and $y(t)$ is the output,
\begin{enumerate}
\item Under what conditions can this system be transformed into a mass-spring string where the input is a force on first mass and output is the position of the first mass.
\item If so, how can the equivalent system (mass and stiffness values) be calculated?
\end{enumerate}
\end{frame}

\begin{frame}{Problem Statement}
In other words: Can we find a coordinate transformation $\mathbf{x} = P \mathbf{q}$, from $\mathbf{q}$ to a new coordinate system $\mathbf{x}$
such that:
%\begin{equation}
%M\ddot{\mathbf{x}}(t) + K\mathbf{x}(t) = \mathbf{\hat{b}}u(t)
%\end{equation}
\begin{equation}
\ddot{\mathbf{x}}(t) + M^{-1} K\mathbf{x}(t) = M^{-1} \mathbf{\hat{b}}u(t)
\end{equation}
\begin{equation}
y(t) = \mathbf{\hat{c}}^T \mathbf{x}(t)
\end{equation}
\footnotesize
$$
M = \begin{bmatrix}
m_1  &  0 & \cdots & 0 \\
0 & m_2  & \ddots & 0 \\
\vdots & \ddots & \ddots & \vdots \\
0 & 0 & \cdots & m_n \end{bmatrix}
, \quad
K = \begin{bmatrix}
k_0+k_1  &  -k_1  & \cdots & 0 \\
-k_1 & k_1+k_2   & \ddots & 0 \\
\vdots & \ddots & \ddots & -k_{n-1} \\
0 & 0  & -k_{n-1} &  k_{n-1} + k_n \end{bmatrix}
$$
\normalsize
$$
\mathbf{\hat{b}} = \begin{bmatrix} 1 & 0 & \cdots & 0 \end{bmatrix}^T
,\quad \mathbf{\hat{c}} = \begin{bmatrix} 1 & 0 & \cdots & 0 \end{bmatrix}^T
$$
\centering
%Styles
\tikzstyle{ground}=[color=gray,postaction={draw,decorate,decoration={border,angle=-45,amplitude=0.2cm,segment length=2mm}}]
\tikzstyle{cart} = [draw, fill=cyan!10, rectangle, minimum height=1cm, minimum width=1.2cm, rounded corners =0.1cm,node distance=2cm]
\tikzstyle{actuator} = [draw=black ,fill=black!10!white, thick, rectangle, inner sep=0,minimum height=0.6cm, minimum width=0.2cm, node distance=3cm]
\tikzstyle{spring} = [thick,black,decorate,decoration={snake,amplitude=3,segment length=10}]
\tikzstyle{wheel} = [thick,orange,decorate,decoration={coil,aspect=0.7,amplitude=5}]
\tikzstyle{cart} = [rectangle, inner sep=0,minimum height=0.8cm, minimum width=1cm, node distance=3cm, path picture={ 
      \shadedraw[left color=white,right color=gray!40!white, thick] ([yshift=0.12cm, xshift=0.3pt] path picture bounding box.south west) rectangle ([xshift=-0.3pt,yshift=-0.3pt] path picture bounding box.north east); 
      \draw[very thick, fill=white] ([yshift=0.12cm, xshift=0.2cm] path picture bounding box.south west) circle (0.1cm);
      \draw[very thick,fill=white] ([yshift=0.12cm, xshift=-0.2cm] path picture bounding box.south east) circle (0.1cm);}]
\tikzstyle{pt} = [coordinate]

\begin{tikzpicture}[scale=1,every node/.style={font=\small}]

%Nodes
	\node[actuator]		(act)					{};
	\node[cart,]	(m1)		[right of=act, node distance=1.4cm]			{$m_1$};
	\node[cart] 	(m2) 		[right of=m1, node distance=1.8cm] 	{$m_2$};
       \node[pt] 	(pt1) 		[right of=m2, node distance=1cm] 	{};
	\node[pt] 	(pt2) 		[right of=pt1, node distance=0.5cm] 	{};
	\node[cart] 	(m3) 		[right of=pt2, node distance=1cm] 	{$m_n$};

%Connections	
	\draw[spring] (act) -- node[above, yshift=2pt] {$k_1$} (m1);
	\draw[spring] (m1) -- node[above, yshift=2pt] {$k_2$} (m2);
	\draw[spring] (m2) -- node[above] {$k_3$} (pt1);
	\draw[thick, dashed] (pt1) -- (pt2);
	\draw[spring] (pt2) -- node[above] {$k_n$} (m3);
	
	\draw[ground] (m1.south west) -- (m1.south east);
	\draw[ground] (m2.south west) -- (m2.south east);
	\draw[ground] (m3.south west) -- (m3.south east);
	
	\draw[gray] ([yshift=0.35cm]act.north) -- +(0,-0.1cm);
	\draw[->,gray] ([yshift=0.3cm]act.north) node[above] (A0) {$x_0$} -- +(0.5cm,0);
	\draw[gray] ([yshift=0.35cm]m1.north) -- +(0,-0.1cm);
	\draw[->,gray] ([yshift=0.3cm]m1.north) node[above] (A1) {$x_1$} -- +(0.5cm,0);
	\draw[gray] ([yshift=0.35cm]m2.north) -- +(0,-0.1cm);
	\draw[->,gray] ([yshift=0.3cm]m2.north) node[above] (A2) {$x_2$} -- +(0.5cm,0);
	\draw[gray] ([yshift=0.35cm]m3.north) -- +(0,-0.1cm);
	\draw[->,gray] ([yshift=0.3cm]m3.north) node[above] (A0) {$x_n$} -- +(0.5cm,0);
	
\end{tikzpicture}

\end{frame}
%%%%% image of mass spring required form

%%%%%%%%%%%%%%%%%%%%%%%%%%%%%%%%%%%%%%%%%%%%%%%%%%%%%%%%%%%%%%%%%%%%%%%%%
%%%%%%%%%%%%%%%%%%%%%%%%%-------METHODS--------%%%%%%%%%%%%%%%%%%%%%%%%%%
%%%%%%%%%%%%%%%%%%%%%%%%%%%%%%%%%%%%%%%%%%%%%%%%%%%%%%%%%%%%%%%%%%%%%%%%%
\section{Methods}
\subsection{Inverse Eigenvalue Problem for Jacobi Matrices}
\begin{frame}{Scaling of input and output vectors}
First consider the transformation $\mathbf{v} = A \mathbf{q}$
$$A = \begin{bmatrix}
a_1  &  0 & \cdots & 0 \\
0 & a_2  & \ddots & 0 \\
\vdots & \ddots & \ddots & \vdots \\
0 & 0 & \cdots & a_n \end{bmatrix}$$
Scales the $\mathbf{b}$ and $\mathbf{c}$ vectors without affecting $\Lambda$
\begin{equation}
\ddot{\mathbf{v}}(t) +  \Lambda \mathbf{v}(t) =  A \mathbf{b}u(t) =  \begin{bmatrix} a_1 b_1 \\ a_2 b_2 \\ \vdots \\ a_n b_n  \end{bmatrix} u(t)
\label{eq:scaled1}
\end{equation}
\begin{equation}
y(t) = \mathbf{c}^T  A^{-1} \mathbf{v}(t) = \begin{bmatrix} c_1/a_1 &  c_2/a_2 & \cdots & c_n/a_n \end{bmatrix} \mathbf{v}(t)
\label{eq:scaled2}
\end{equation}
\end{frame}

\begin{frame}{Inverse eigenvalue problem for Jacobi matrix}
Jacobi matrix:
$$J =  M^{-\frac{1}{2}} K M^{-\frac{1}{2}}$$
\vspace{0.5cm}
$$
J = \begin{bmatrix} d_1  &  -e_1 & 0 & \cdots & 0 \\
-e_1 & d_2  & -e_2 & \ddots & 0 \\
0 & -e_2 & \ddots & \ddots & \vdots \\
\vdots & \ddots & \ddots & d_{n-1} & -e_{n-1} \\
0 & 0 & \cdots & -e_{n-1} & d_n \end{bmatrix}
$$

Inverse eigenvalue problem:

Find $X$ such that $$J = X^{-1} \Lambda X$$
\end{frame}

\begin{frame}{Solution of the inverse eigenvalue problem}
\begin{itemize}
\item There is a unique Jacobi matrix $\mathbf{J}$ having specified eigenvalues $(\lambda_i)_1^n$, where
$$
0 \leq \lambda_1<\lambda_2< \cdots <\lambda_n
$$
and with normalised eigenvectors $(\mathbf{u}_i)_1^n$  having non-zero specified values $(u_{1i})_1^n$ of their first components \footcite{gladwell1986inverse}.
\item If $U$ is the matrix whose columns are the normalised eigenvectors then let $X = U^T$.
\item Then the theorem says that for a given $\Lambda$ and $\mathbf{x_1}$ (the first column of $X$) there is a unique Jacobi matrix $J$ and orthogonal matrix $X$ such that $$J = X^T \Lambda X$$.
\end{itemize}
\end{frame}

\begin{frame}{Solution of the inverse eigenvalue problem}
The Lanczos algorithm may be used to calculate the matrices $J$ and $X$ for a given $\Lambda$ and $\mathbf{x_1}$
\vspace{0.5cm}
Using this algorithm we can find the second part of the transformation $\mathbf{z} = X^T \mathbf{v}$ which leads to the new system:
\begin{equation}
\ddot{\mathbf{z}}(t) + J \mathbf{z}(t) =   U A^{-1} \mathbf{b} u(t)
\end{equation}
\begin{equation}
y(t) = \mathbf{c}^T  A U^T \mathbf{z}(t)
\end{equation}
\end{frame}

\begin{frame}{Reconstruction of Mass and Stiffness matrices}
Reconstruct the mass and stiffness matrices $M$ and $K$ from $J$
$$K = M^{\frac{1}{2}} J M^{\frac{1}{2}}$$

$k_0 = k_n = 0$ (free-free system)

\vspace{0.5cm}
The rows of $K$ must sum to zero:
\begin{equation}
K \begin{bmatrix} 1 & 1 & \cdots & 1 \end{bmatrix}^T =  M^{\frac{1}{2}} J M^{\frac{1}{2}} \begin{bmatrix} 1 & 1 & \cdots & 1 \end{bmatrix}^T = \mathbf{0}
\end{equation} 
\begin{equation}
J \begin{bmatrix} \sqrt{m_1} & \sqrt{m_2} & \cdots & \sqrt{m_n} \end{bmatrix}^T = \mathbf{0}
\end{equation}
Choose a value for $m_1$
Solve row by row to calculate each $m_i$ and $k_i$
\end{frame}

\begin{frame}{Final system}
The transformation $\mathbf{x} = M^{\frac{1}{2}} \mathbf{z}$ leads to the final system:
\begin{equation}
M \ddot{\mathbf{x}}(t) + K \mathbf{x}(t) = M^{\frac{1}{2}} U A^{-1} \mathbf{b} u(t)
\label{eq:fin1}
\end{equation}
\begin{equation}
y(t) = \mathbf{c}^T  A U^T M^{\frac{1}{2}}\mathbf{x}(t)
\label{eq:fin2}
\end{equation}
By combining the three steps the overall transformation can be written as:
\begin{equation}
\mathbf{x} = P \mathbf{q}
\end{equation}
\begin{equation}
P =  M^{-\frac{1}{2}} U A^{-1}
\label{eq:p}
\end{equation}
\end{frame}

\subsection{Correct for of input and output vectors}
\begin{frame}{Achieving the desired $\mathbf{b}$ and $\mathbf{c}$ vectors}
\begin{equation}
M^{\frac{1}{2}} U A^{-1} \mathbf{b} = M^{\frac{1}{2}} U A \mathbf{c} = \begin{bmatrix} 1 &  0 & \cdots & 0 \end{bmatrix}^T
\end{equation}
\begin{equation}
A^{-1} \mathbf{b} = A \mathbf{c} = \frac{1}{m_1} U^T \begin{bmatrix} 1 &  0 & \cdots & 0 \end{bmatrix}^T
\end{equation}
\begin{equation}
A^{-1} \mathbf{b} = A \mathbf{c} = \frac{1}{m_1} \mathbf{x}_1
\label{eq:bc2}
\end{equation}
Solution:
\begin{equation}
a_i = \sqrt{\frac{b_i}{c_i}} ,\quad m_1 = \sqrt{\frac{1}{\sum b_i c_i}} ,\quad x_{i1} = \sqrt{\frac{b_i c_i}{\sum b_i c_i}}
\label{eq:bcxm}
\end{equation}
Only possible if all the $b_i c_i$ have the same sign
\end{frame}

%%%%%%%%%%%%%%%%%%%%%%%%%%%%%%%%%%%%%%%%%%%%%%%%%%%%%%%%%%%%%%%%%%%%%%%%%
%%%%%%%%%%%%%%%%%%%%%%%%%-------RESULTS--------%%%%%%%%%%%%%%%%%%%%%%%%%%
%%%%%%%%%%%%%%%%%%%%%%%%%%%%%%%%%%%%%%%%%%%%%%%%%%%%%%%%%%%%%%%%%%%%%%%%%
\section{Results}

\subsection{Segmented Rocket Model}

\begin{frame}{Segmented rocket model}
\begin{columns}
\column{0.5\textwidth}
\centering
\input{graphics/seg3-rocket-allact-pres.tex} \\
Segmented rocket model
\column{0.5\textwidth}
\begin{itemize}
\item 3 rigid bodies, 2 torsional springs
\item 4 actuators - three lateral thruster forces $f_i$
\item gimballed engine thrust - $\delta$
\item Three outputs - attitude sensors on each segment - $\alpha_1$, $\alpha_2$ and $\alpha_3$
\end{itemize}
\end{columns}
\end{frame}

\begin{frame}{Segmented rocket model}
\begin{itemize}
\item Equations of motion:
\begin{multline}
\left[\begin{matrix}3 I + 2 h^{2} m & 2 I + \frac{3 m}{2} h^{2} & I + \frac{h^{2} m}{2}\\2 I + \frac{3 m}{2} h^{2} & 2 I + \frac{7 m}{6} h^{2} & I + \frac{5 m}{12} h^{2}\\I + \frac{h^{2} m}{2} & I + \frac{5 m}{12} h^{2} & I + \frac{h^{2} m}{6}\end{matrix}\right]\left[\begin{matrix}\ddot{\theta}\\\ddot{\phi}_{1}\\\ddot{\phi}_{2}\end{matrix}\right] + \left[\begin{matrix}0 & - \frac{2 T}{3} h & - \frac{T h}{6}\\0 & - \frac{2 T}{3} h + k & - \frac{T h}{6}\\0 & - \frac{T h}{6} & - \frac{T h}{6} + k\end{matrix}\right]\left[\begin{matrix}\theta\\\phi_{1}\\\phi_{2}\end{matrix}\right]\\ = \left[\begin{matrix}\frac{3 T}{2} h & \frac{h}{2} & - \frac{h}{2} & - \frac{3 h}{2}\\\frac{2 T}{3} h & \frac{2 h}{3} & - \frac{h}{3} & - \frac{4 h}{3}\\\frac{T h}{6} & \frac{h}{6} & \frac{h}{6} & - \frac{5 h}{6}\end{matrix}\right]\left[\begin{matrix}\delta\\f_{1}\\f_{2}\\f_{3}\end{matrix}\right]
\end{multline}
\item Output equation:
\begin{equation}
\left[\begin{matrix}\alpha_{1}\\\alpha_{2}\\\alpha_{3}\end{matrix}\right] = \left[\begin{matrix}1 & 0 & 0\\1 & 1 & 0\\1 & 1 & 1\end{matrix}\right]\left[\begin{matrix}\theta\\\phi_{1}\\\phi_{2}\end{matrix}\right]
\end{equation}
\item Choice of 4 actuators and 3 attitude sensors
\end{itemize}
\end{frame}

\begin{frame}{Model Parameters}
\begin{columns}
\column{0.3\textwidth}
\centering
\includegraphics[height=0.75\textheight]{images/Vega.png} \\
Vega Rocket
\column{0.7\textwidth}
\begin{itemize}
\item Numerical values for model parameters representative of European \emph{Vega} rocket \footcite{Perez2006}
\item Chosen to match MOI, mass and first two vibration frequencies
\end{itemize}
\begin{table}
  \begin{center}
    \begin{tabular}{ ccc }
	\hline
           Parameter & Value & Unit \\
	\hline
      	$m$ & $5 \cdot 10^4$ & $kg$\\
      	$k$ & $6 \cdot 10^7$ & $Nm$\\
      	$I$ & $5 \cdot 10^5$ & $kg~m^2$\\
      	$h$ & $10$ & $m$\\
      	$T$ & $2.3 \cdot 10^6$ & $N$\\
    \end{tabular}
  \end{center}
\end{table}
\end{columns}
\end{frame}

\begin{frame}{Model Parameters}
\begin{itemize}
\item Diagonalised equations of motion
\item 4 possible actuators, 3 possible sensors
\end{itemize}
\begin{multline}
\left[\begin{matrix}\ddot{q}_{1}\\\ddot{q}_{2}\\\ddot{q}_{3}\end{matrix}\right] + \left[\begin{matrix}0 & 0 & 0\\0 & 61.3 & 0\\0 & 0 & 309.0\end{matrix}\right]\left[\begin{matrix}q_{1}\\q_{2}\\q_{3}\end{matrix}\right] = \left[\begin{matrix}2.73 & 5.08 \cdot 10^{-7} & -4.08 \cdot 10^{-7} & -1.39 \cdot 10^{-6}\\-14.7 & 3.19 \cdot 10^{-6} & 3.19 \cdot 10^{-6} & -6.38 \cdot 10^{-6}\\19.9 & -1.05 \cdot 10^{-5} & 1.06 \cdot 10^{-5} & -8.93 \cdot 10^{-6}\end{matrix}\right]\left[\begin{matrix}\delta\\f_{1}\\f_{2}\\f_{3}\end{matrix}\right]
\label{eq:num_rock1}
\end{multline}
\begin{equation}
\left[\begin{matrix}\alpha_{1}\\\alpha_{2}\\\alpha_{3}\end{matrix}\right] = \left[\begin{matrix}1.0 & -0.592 & 0.322\\1.0 & -0.00814 & -0.347\\1.0 & 0.547 & 0.322\end{matrix}\right]\left[\begin{matrix}q_{1}\\q_{2}\\q_{3}\end{matrix}\right]
\label{eq:num_rock2}
\end{equation}
\end{frame}

\subsection{Test Cases for different sensor-actuator combinations}

\begin{frame}{Transformed Systems}
\begin{columns}
\column{\textwidth}
    \includegraphics[width=0.85\textwidth]{graphics/rocket-trans-pres.pdf}
\end{columns}
\end{frame}

\begin{frame}{Example Step Response}{Test Case 1}
\begin{center}
    \input{graphics/testcase1.tex}
\end{center}
\end{frame}

\begin{frame}{Example Step Response}{Test Case 2}
\begin{center}
    \input{graphics/testcase3.tex}
\end{center}
\end{frame}

\begin{frame}{Example Step Response}{Test Case 3}
\begin{center}
    \input{graphics/testcase2.tex}
\end{center}
\end{frame}

%%%%%%%%%%%%%%%%%%%%%%%%%%%%%%%%%%%%%%%%%%%%%%%%%%%%%%%%%%%%%%%%%%%%%%%%%%%%
%%%%%%%%%%%%%%%%%%%%%%%%%-------SUMMARY--------%%%%%%%%%%%%%%%%%%%%%%%%%%
%%%%%%%%%%%%%%%%%%%%%%%%%%%%%%%%%%%%%%%%%%%%%%%%%%%%%%%%%%%%%%%%%%%%%%%%%%%%
\section{Summary}

\begin{frame}{Summary}
  \begin{itemize}
  \item
  Way to transform a given SISO system to the equivalent mass-spring representation where the input is a force on the first mass and the output is the position of the first mass.
  \item
    Two conditions
	\begin{itemize}
	\item Non-negative and distinct eigenvalues: $0 \leq \lambda_1<\lambda_2< \cdots <\lambda_n$
	\item Form of input and output vectors: $b_i c_i>0 \quad \forall i$ or $b_i c_i<0 \quad \forall i$
	\end{itemize}
  \item New way of looking at the control problem: non-uniformity of the resulting mass-spring system and the position of this non-uniformity indicate the difficulty of the control problem.
  \item On multibody systems, choice of actuator and sensor affect the resulting mass-spring system.
  \item
    Future Work
    \begin{itemize}
    \item Equivalent systems where the output may be the position of a different mass.
    \item Optimisation of actuator and sensor locations.
    \end{itemize}
  \end{itemize}
\end{frame}

\begin{frame}{}
\centering
\huge{Thank You!}
\vspace{0.8cm}

\huge{Questions?}
\end{frame}




% All of the following is optional and typically not needed. 
\appendix
\section<presentation>*{\appendixname}
\subsection<presentation>*{For Further Reading}

\begin{frame}[allowframebreaks]
  \frametitle<presentation>{For Further Reading}
\printbibliography
\end{frame}

\end{document}
