\documentclass{mbd_fullpaper}

\begin{document}

%New commands
\newcommand{\heading}[1]{
   {\medskip\hskip5em\bf\large{#1}
   \vskip0.5ex
   }
}
\newcommand{\eqnref}[1]{
  (\ref{#1})
}

\renewcommand{\refname}{\medskip\bf\large References}



%------------------------------------------------------------
% title
\begin{center}
  \Large{\bf
Equivalent Mass-Spring Models of Multibody Spacecraft for the Application of Wave-based Control  }
\end{center}

%------------------------------------------------------------
% authors
\begin{center}
\large{
Joseph Thompson
}
\end{center}

%------------------------------------------------------------
% affiliations
{
\begin{center}
 \small
  \begin{tabular}{c}
    School of Mechanical and Materials Engineering \\
    University College Dublin              \\
    Belfield, Dublin 4, Ireland        \\
    joseph.thompson@ucdconnect.ie                        \\
  \end{tabular}
\end{center}
}

%------------------------------------------------------------
% abstract



\section*{Abstract}

\section{Introduction}

Many mechanical systems are underactuated i.e. have more degrees of freedom than actuators.
In attempting to move such a system, one will usually excite certain modes of vibration.
To bring the system back to rest, the control algorithm must dampen these vibrations.
To end up in the correct target position at the end of a manoeuvre this exciting and damping of vibrations must be done in precisely the right way.
One possible approach is to model the motion of each part of the system as a superposition of two components or `waves'.
One wave is travelling from the actuator into the system and the other is travelling from the system back towards the actuator.
In this way the actuator is simultaneously launching a wave into the system and absorbing a returning wave from the system.
One complete manoeuvre consists of launching a certain wave into the system and then absorbing the returning wave to bring the system back to rest.
If the initial launch wave is correct then the final position will also be correct.

LR of WBC
Modelling motion of mass-spring strings as waves.
Two types of systems
Linear control system measures returning wave using first and second mass/ force
Disturbances, MIMO, measuring different masses

These contol systems have been applied and tested numerically on a much wider class of systems.
However, in ms systems the relationship to an underlying wave model is well understood, not for these systems.

The cascaded system of Fig.~\ref{fig:rec_sys} is similar to a range of systems of practical engineering interest, including robot arms, cranes, space structures and disk drive heads. Much work has been done on wave-based modelling and control of such systems \cite{OConnor2011, Connor2005}. This paper asks the question: to what extent can this work be extended to a wider class of systems? 
This question is motivated by the control of spacecraft with features such as structural flexibility, flexible appendages and fuel slosh.
Mathematical models of these systems, as presented to the control engineer, do not appear to have the same structure as a mass-spring string actuated at one end.
Many systems for instance are modelled as multi-rigid-body systems ~\cite{Kane1980} with multiple sensors and actuators or simply presented as a linear system in the form of a set of state space matrices.
This leads to the following problem. Given a generic, SISO (single-input single-output), lightly damped mechanical system described by
\begin{equation}
\ddot{\mathbf{q}}(t) + \Lambda\mathbf{q}(t) = \mathbf{b}u(t)
\label{eq:modal1}
\end{equation}
\begin{equation}
y(t) = C \mathbf{q}(t)
\label{eq:modal2}
\end{equation}
\begin{equation}
\Lambda = \begin{bmatrix}
\lambda_1  &  0 & \cdots & 0 \\
0 & \lambda_2  & \ddots & 0 \\
\vdots & \ddots & \ddots & \vdots \\
0 & 0 & \cdots & \lambda_n \end{bmatrix}
\end{equation}
under what conditions can this system be transformed into the equivalent of a mass-spring string such as in Fig.1 (for which WBC is known to work well)?
Here $u(t)$ is the input and $y(t)$ is the output.
We restrict ourselves to the case where the eigenvalues are real and distinct, i.e.
\begin{equation}
0 \leq \lambda_1<\lambda_2< \cdots <\lambda_n
\label{eq:lambda}
\end{equation}
The problem becomes if (and if so, how) this system can be transformed to a structure similar to that of the system in Fig.1.
The required structure is
\begin{equation}
M\ddot{\mathbf{z}}(t) + K\mathbf{z}(t) = \mathbf{\hat{b}}u(t)
\label{eq:eom1}
\end{equation}
\begin{equation}
y(t) = \hat{C} \mathbf{z}(t)
\label{eq:eom2}
\end{equation}
\begin{equation}
M = \begin{bmatrix}
m_1  &  0 & \cdots & 0 \\
0 & m_2  & \ddots & 0 \\
\vdots & \ddots & \ddots & \vdots \\
0 & 0 & \cdots & m_n \end{bmatrix}
, \quad
K = \begin{bmatrix}
k_1+k_2  &  -k_2 & 0 & \cdots & 0 \\
-k_2 & k_2+k_3  & -k_3 & \ddots & 0 \\
0 & -k_3 & \ddots & \ddots & \vdots \\
\vdots & \ddots & \ddots & k_{n-1}+k_n & -k_{n} \\
0 & 0 & \cdots & -k_{n} & k_n \end{bmatrix}
\end{equation}
An algorithm is developed to find a suitable coordinate transformation $\mathbf{z} = P \mathbf{q}$ to achieve this objective.
The problem may be first reduced to an inverse eigenvalue problem for a Jacobi matrix. This may be solved using the Lanczos algorithm as presented in \cite{gladwell1986inverse}. 
It is found that there are many equivalent mass-spring systems depending on the desired forms of the input vector $\hat{\mathbf{b}}$ and output matrix $\hat{C}$, that is, on the input-output structure of the system or the location(s) of actuators and sensors in the string.



Road map
Section 2 - WBC control system
Section 3 - Mathematical model of segmented rocket
Section 4 - Class of systems to which WBC can be applied, and algorithm for converting to standard mass-spring form
Section 5 - Numerical Examples - different position of sensors, actuators
Section 6 - Discussion
Section 7 - Conclusions 

\section{Wave-Based Control Scheme}
The details of the wave-based control scheme

\section{Mathematical Model of Segmented Rocket}

Note not the same as mass-spring
The particular example we use to test the algorithm is a planar model of a rocket with a flexible structure as in Fig.~\ref{fig:flex_rocket}. The model consists of three rigid bodies connected by two torsional springs. The base body has an attitude angle $\theta$ relative to an inertial reference frame and the other bodies have relative angles $\phi_1$ and $\phi_2$ respectively between themselves and the segment below as shown. The rocket has a gimballed engine producing a thrust $T$ at an angle $\delta$ to the main body and also a lateral thruster located on the bottom segment which produces a variable thrust $f$. Different test cases are examined where different actuators are used and attitude sensors are located at different positions along the rocket body i.e. on different segments. In each case an equivalent mass-spring model for the system is calculated.

\begin{equation}
\input{python/eqM.tex}
\end{equation}

\section{Class of systems suitable for WBC}

\section{Numerical Examples}
Several models
Physical parameters vega

\section{Discussion}
Is conversion possible

Mass Spring ratios, uniformity
Wave-based intuition
Longer settling time for actuator at top

\section{Conclusions}

\section{ext abs}

% 250 words max.
Wave-Based Control (WBC) is particularly effective for achieving rest to rest motion of under-actuated, cascaded, lumped flexible systems.
In this control scheme the actuator simultaneously launches and absorbs wave components travelling into and out of the system at one end.
By doing this the control scheme combines position control and active vibration damping.
Much work has been done on wave-based modelling and control of mass-spring strings.
This paper asks the question: to what extent can this work be extended to a wider class of systems?
This question is motivated by the control of spacecraft with features such as structural flexibility, flexible appendages and fuel slosh.
Most mathematical models of these systems presented to the control engineer, do not obviously have the structure of a mass-spring string.
However, often it is possible to calculate an equivalent mass-spring system.
This paper identifies a class of systems, with real, positive and distinct eigenvalues, for which this transformation is possible and presents an algorithm for calculating the equivalent mass and spring values.
A segmented planar multibody rocket model is used as an example.
This model consists of three rigid bodies connected by two torsional springs and a gimballed rocket engine with constant thrust which may be used for attitude control.
Several test cases with different sensor and actuator configurations are examined and equivalent mass-spring systems are calculated in each case.


%------------------------------------------------------------
% biliography
\bibliographystyle{acm}
\bibliography{eccomas_2017}

\end{document} 
